% Dokumentformat
\documentclass[12pt, a4paper]{scrreprt}
%\documentclass[12pt, a4paper]{scrreprt}

% Zeichensatz festlegen
\usepackage[utf8]{inputenc} %Linux
% \usepackage[latin1]{inputenc} %Windoof
\usepackage[T1]{fontenc}
\usepackage[ngerman]{babel}

% Damit Text ``gesperrt'' werden kann
\usepackage{soul}

% Pakete fuer Grafiken
\usepackage{graphicx}

% Pakete fuer Tabellen
\usepackage{color,colortbl}
\usepackage{booktabs}

% Paket fuer Kopf und Fusszeilen
\usepackage{fancyhdr}
\pagestyle{fancy}

% remember chapter number and title
\renewcommand{\chaptermark}[1]%
	{\markboth{#1}{}}
\renewcommand{\sectionmark}[1]%
	{\markright{\thesection\ #1}}

% positioning
\lhead[\fancyplain{}{\bfseries\thepage}]%
	{\fancyplain{}{\bfseries\rightmark}}
\rhead[\fancyplain{}{\bfseries\leftmark}]%
	{\fancyplain{}{\bfseries\thepage}}
\cfoot{}

\fancypagestyle{plain}{%

\lhead[\fancyplain{}{\bfseries\thepage}]%
	{\fancyplain{}{}}

\rhead[\fancyplain{}{}]%
	{\fancyplain{}{\bfseries\thepage}}
}

% Einruecken verhindern
\parindent 0pt

% Leerzeile nach Absatz einfuegen
\parskip 12pt

% Zeilenabstaende
\usepackage{setspace}

% Für PDF-Header
\usepackage[
	bookmarks=true,
	bookmarksopen=true,
	bookmarksnumbered=true,
	breaklinks=true,
	colorlinks=true,
	linkcolor=black,
	anchorcolor=black,
	citecolor=black,
	filecolor=black,
	menucolor=black,
	urlcolor=black
]{hyperref}

\hypersetup{
	pdfstartpage=1,
	pdftitle={Zusammenfassung - Bachelorarbeit},
	pdfauthor={Vorname Nachname},
	pdfsubject={Titel der Arbeit},
	pdfkeywords={Bachelorarbeit, Zusammenfassung},
}