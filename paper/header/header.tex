% Dokumentformat
\documentclass[12pt, a4paper]{scrreprt}
%\documentclass[12pt, a4paper]{scrreprt}

% Zeichensatz festlegen
\usepackage[utf8]{inputenc} %Linux
% \usepackage[latin1]{inputenc} %Windoof
% \usepackage[T1]{fontenc}
% \usepackage[ngerman]{babel}
\usepackage[english]{babel}

% Damit Text ``gesperrt'' werden kann
\usepackage{soul}

\usepackage{float}

% Paket fuer Grafiken
\usepackage{graphicx}

% Pakete fuer Tabellen
% \usepackage{color,colortbl}
\usepackage{booktabs}

% Paket fuer Kopf und Fusszeilen
\usepackage{fancyhdr}
\pagestyle{fancy}

% remember chapter number and title
\renewcommand{\chaptermark}[1]%
	{\markboth{#1}{}}
\renewcommand{\sectionmark}[1]%
	{\markright{\thesection\ #1}}

% positioning
\lhead[\fancyplain{}{\bfseries\thepage}]%
	{\fancyplain{}{\bfseries\rightmark}}
\rhead[\fancyplain{}{\bfseries\leftmark}]%
	{\fancyplain{}{\bfseries\thepage}}
\cfoot{}

\fancypagestyle{plain}{%

\lhead[\fancyplain{}{\bfseries\thepage}]%
	{\fancyplain{}{}}

\rhead[\fancyplain{}{}]%
	{\fancyplain{}{\bfseries\thepage}}
}

% Einruecken verhindern
\parindent 0pt

% Leerzeile nach Absatz einfuegen
\parskip 12pt

% Zeilenabstaende
\usepackage{setspace}

% Roemische Zahlen
\newcommand{\RM}[1]{\MakeUppercase{\romannumeral #1{}}}

% Bibtex
\usepackage{bibgerm}
%\usepackage{babelbib}

% Abkürzungsverzeichnis
\usepackage[printonlyused]{acronym}

% Bilder nebeneinander, ...auskommentiert wegen \subcaption
\usepackage{subfig}

% Abbildungsverzeichnistiefe einstellen
\captionsetup{lofdepth=4}

% Eurosymbol
\usepackage[right]{eurosym}

% Fussnoten ueber alle kapitel zaehlen
\usepackage{chngcntr}
\counterwithout{footnote}{chapter}

% Listings
% \usepackage{listings}
% \usepackage{xcolor}
% \definecolor{hellgrau}{rgb}{0.9,0.9,0.9}
% \definecolor{colKeys}{rgb}{0,0,1}
% \definecolor{colIdentifier}{rgb}{0,0,0}
% \definecolor{colComments}{rgb}{1,0,0}
% \definecolor{colString}{rgb}{0,0.5,0}
% \lstset{
%     float=hbp,
%      basicstyle=\ttfamily\color{black}\small,
% % %     basicstyle=\ttfamily\color{black}\small\smaller,
%     identifierstyle=\color{colIdentifier},
%     keywordstyle=\color{colKeys},
%     stringstyle=\color{colString},
%     commentstyle=\color{colComments},
%     columns=flexible,
%     tabsize=2,
%     frame=single,
%     extendedchars=true,
%     showspaces=false,
%     showstringspaces=false,
%     numbers=none,
%     numberstyle=\tiny,
%     breaklines=true,
%     backgroundcolor=\color{hellgrau},
%     breakautoindent=true
% }

% Für PDF-Header
\usepackage[
	bookmarks=true,
	bookmarksopen=true,
	bookmarksnumbered=true,
	breaklinks=true,
	colorlinks=true,
	linkcolor=black,
	anchorcolor=black,
	citecolor=black,
	filecolor=black,
	menucolor=black,
	urlcolor=black
]{hyperref}

\hypersetup{
	pdfstartpage=1,
	pdftitle={Bachelorarbeit},
	pdfauthor={Vorname Nachname},
	pdfsubject={Titel der Arbeit},
	pdfkeywords={Bachelorarbeit},
}

% mehrere Literaturverzeichnisse
\usepackage{multibib}

% Definition des citebefehls für Literatur
%\newcites{lit}{Bibliography}
% Definition des citebefehls für Internet
\newcites{int}{Literature}

% ############### Weiteres Zeug für moi #######################
% schöne autotabellen
\usepackage{booktabs}

% Farben! 
\usepackage{xcolor}

% Mathe Zeichen
\usepackage{amsmath}
\usepackage{amssymb}
\newcommand{\R}{\mathbb{R}}

% hyperrefs
\usepackage{hyperref}

% svg 
%\usepackage{svg}

% For subplots in plots?
% \usepackage{subcaption}