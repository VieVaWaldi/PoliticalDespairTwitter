Sentimentalitätsanalysen sind ein mächtiges Mittel, um große Mengen an Text zu quantifizieren und zu kategorisieren. Diese werden von vielen Instituten verwendet um beispielsweise das politische Klima zu Themen wie Pandemien, der Klimaerwärmung oder Krieg zu analysieren \citeint{mitSent}. Ergebnisse dieser Arbeit sind allerdings kritisch zu bewerten. Besonders das Berechnen der Sentimentalität ist in dieser Analyse abhängig von \textit{TextBlob}. Durch ein fehlendes Validierungsdatenset sowie fehlende Labels, die Aufschluss über die Grundwahrheit der verwendeten Daten gegeben, ist es schwer, die Genauigkeit der Analyse festzustellen. Auch in Betracht dessen, das nicht klar ist mit welchen Daten die Bibliothek trainiert wurde. Shahul ES. hat in seinem Blog \textit{Sentiment Analysis in Python: TextBlob vs Vader Sentiment vs Flair vs Building It From Scratch} versucht TextBlob und zwei andere Sprachverarbeitungsbibliotheken zu vergleichen und ist für die hier verwendete Biblithek auf einen Genauigket von 56\% gekommen \citeint{sahulEs}.

Weiterhin lassen die Ergebnisse, die zu Beginn gegebene Hypothese von Michael Dimock et. al. aus dem Paper \textit{Political Polarization in the American Public} von 2014 nicht bekräftigen \label{polPolar} \citeint{politicalPolarization}. Dies liegt zum einen an der verwendeten Stichprobe. Während in dem Paper Amerikaner*innen aus jedem Metier befragt wurden, befasst sich diese Analyse aufgrund von Datenbeschaffungsproblemen ausschließlich mit Politikern*innen. Abgesehen davon sind in dieser Stichprobe wesentlich weniger Menschen vertreten was die Analyse weiterhin verzerrt. Zum anderen werden hier weniger Metriken verglichen, lediglich die zu Beginn kritisch zu bewertende Subjektivität und Polarität aus der Sprachverarbeitungsbibliothek.

In Anbetracht dessen, liefert diese Arbeit ein grundlegendes Werkzeug um Daten aus Twitter zu verarbeiten. In einem weitern Verlauf wären mehr Daten von mehreren unterschiedlichen Usern aus den Vereinigten Staaten eine Möglichkeit um diese Analyse zu erweitern. Dafür sind lediglich die Twitternamen von diesen Usern erforderlich welche ebenfalls durch Scraping erlangt werden können. Zu beachten ist die moralische Fragwürdigkeit dieses Prozess weswegen unter anderem vorerst darauf verzichtet wurde. Schließlich kann die eigentliche Verarbeitung der Sentimentalität durch die die implementierte Schnittstelle leicht ausgetauscht werden, um so ebenfalls eine genauere oder zumindest andere Analyse im Vergleich durchzuführen.
