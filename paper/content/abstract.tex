\newenvironment{changemargin}[2]{%
\begin{list}{}{%
\setlength{\topsep}{0pt}%
\setlength{\leftmargin}{#1}%
\setlength{\rightmargin}{#2}%
\setlength{\listparindent}{\parindent}%
\setlength{\itemindent}{\parindent}%
\setlength{\parsep}{\parskip}%
}%
\item[]}{\end{list}}

\begin{abstract}

\begin{changemargin}{2cm}{2cm}
  
{\centering \textbf{Abstract} \par}
\medskip

In der vorliegenden Arbeit behandeln wir eine Sentimentalitätsanalyse von US amerikanischen Politikern aus dem \textit{'House of Representatives'}. Dazu haben wir Daten von Twitter der letzten 12 Jahren zu den genannten Repräsentatnten \textit{gescrapt} und mit Hilfe von des Big Data Frameworks Spark verarbeitet. Ziel der Sentimentalitätsanalyse war es Unterschiede der beiden Parteien (Republikaner und Demokraten) zu bestimmten politischen und auch allgemeinen Themen herauszufiltern. Jedoch haben sich in den gegebenen Daten weniger Diskrepanzen zwischen den beiden Parteien erkennen lassen, als zu Beginn erwartet, wie im Laufe dieser Arbeit verdeutlicht wird.

\end{changemargin}

\end{abstract}