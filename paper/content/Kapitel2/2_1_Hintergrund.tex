\section{Background}

	- Allgemeine Vorstellung der Packages ihr Vor und Nachteile
	
	\subsection{GetOldTweets3-Pakage}
	%\begin{Absatz}
	- Was ist der Vorteil der Scraping Bibliothek und Nachteile
	
	GetOldTweets3 ist ein kostenlose Python 3 Packages mit welchen Twitterdaten ohne API-Schlüssel abgerufen werden können.
	Mit GetOldTweets3 können Sie Tweets mit einer Vielzahl von Suchparametern wie Start-/Enddatum, Benutzername(n), Textabfrage 
	und Referenzortbereich durchsuchen. Außerdem können Sie angeben, welche Tweet-Attribute Sie einbeziehen möchten. Einige Attribute
	sind: Nutzername, Tweettext, Datum, Retweets und Hashtags.[1]
	Die offizielle API von Twitter hat eine lästige Zeiteinschränkung, weshalb man keine Tweets älter als eine Woche abrufen 
	kann. Es gibt einige Tools die Zugang zu älteren Tweets anbieten, diese sind jedoch meistens kostenpflichtig. Das Forscher-
	team hat nach einem andere Tool gesucht die diese Aufgaben übernehmen, wodurch die Wahl auf das Package GetOldTweets3 gefallen 
	ist.[2]   	
	Die Analyse des Codes von GetOldTweets3 und die Funktionsweise des Searchthrough Browsers von Twitter zeigt wie das Packages auch
	an alte Tweets kommt. Grundsätzlich, wenn sie auf Twitter seiten eingeben oder User suchen, startet ein Scroll-Loader. D.h. wenn sie
	dann weiter nach unten scrollen, bekommen sie immer mehr Tweets zu den Suchbegriffen. Diese ganzen tweets bekommen sie durch Abfragen 
	an einen JSON-Provider.  
	
	
	Quellen bis jetzt:
	[1]https://andrea-yoss.medium.com/getoldtweets3-830ebb8b2dab ,
	[2]https://pypi.org/project/GetOldTweets3/,
	[3]https://github.com/Jefferson-Henrique/GetOldTweetspython/blob/master/got3/manager/TweetManager.py	

	- GetOldTweets zieht sich die die Abgespeicherten Json-Datein zu einem bestimmten nutzer oder anderen Vorgaben, welche dem Package übergeben werden kann
	- Baut einen User-Agenden, welcher dann die Daten zu der zusammengebauten URL
	
	%\end{Absatz}

	\subsection{NLTK-Natural language Toolkit}
	%\begin{Absatz}
	- Vllt Klären warum wir nicht NLTK verwendet haben, sondern Textblob
	- Vorteile von Textblob gegenüber NLTK
	- Nachteile von Nltk	
	
	%\end{Absatz}

	\subsection{TextBlob}
	%\begin{Absatz}
	- Textblob: Vllt. herausfinden wie die Berechnung stattgefunden hat	
	- Vorteile von Textblob gegenüber NLTK
	- Warum haben wir das Tool genutzt.	
	
	%\end{Absatz}

	\subsection{Sparks}
	%\begin{Absatz}
	- Aus der Vorlesung Vorteile von Spark finden und einbauen	
	
	%\end{Absatz}


	
